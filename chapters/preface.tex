\begin{refsection}
\addchap{Preface}
\largerpage
With this work, we present a more expansive dictionary of the Dagaare language than previously available, paired with a grammatical sketch. We have had to make some choices in putting together this dictionary, and the dictionary should not be seen as representing the whole of the Dagaare language in all its variation.   Languages are always changing, and this dictionary presents the vocabulary of a particular variety of Dagaare at a particular time.  

Dagaare is spoken by a large number of people (one million speakers according to \citealt{Ethnologue}, although this may be conservative, see Section \ref{sec:dagaarestudies}) and throughout a large area.  Dagaare constitutes a dialect continuum which covers mainly the Upper West Region of Ghana. Varieties of the language can be found in the Savanna Region of Ghana. The language is also found in Burkina Faso where it is known mainly as  \textit{Dagara}.
%More properly speaking, Dagaare is not a single language, but constitutes a dialect continuum which traverses the Upper-West  and Northern regions of Ghana into Southern Burkina Faso, where it is named.  
The Dagaare Language Committee \citep{DagaareLanguageCommission1982} established  orthographical conventions and made broad dialect divisions into Northern, Central and Southern Dagaare.  Although these dialects are mutually intelligible to varying degrees, there is a high degree of variation, even from village to village \citep[2-5]{Bodomo1997}.   

This dictionary is primarily concerned with the vocabulary of Central Dagaare, spoken around the area of the towns Jirapa and Ullo.  This dialect has served as the standard dialect for educational materials in Ghana, church literature and radio broadcasts, and we follow that tradition here.   Central Dagaare is also the basis for most linguistic analysis on Dagaare, including the early work of \citet{Wilson1962}, \citet{Kennedy1966} and \citet{Hall1977}.  Southern Dagaare (also known as Waale) is centered around the towns of Wa and Kaleo and widely used as a trade language throughout the region \citep{Bodomo1997}. While the Central and Southern dialects overlap to a substantial degree, there are certainly many differences, some of which are noted in this dictionary.  Two other relevant dialects lying to the West are Southern Birifor, which is mutually intelligible with Southern Dagaare, and Safaliba (population 4000) near Bole in the Northern Region, which, according to \citet{Ethnologue}, has a large lexical similarity to Southern Dagaare.  It is our hope that the publication of this dictionary will serve as a first step to documenting the diversity of the Dagaare language, and will bring to light lexical and grammatical variation across the varieties of Dagaare.    

To illustrate, we list several concepts which are expressed by different lexical items in different dialect zones in Table \ref{tab:dialects}.

\begin{table}
    \begin{tabularx}{\textwidth}{Xllll}\\
    \lsptoprule 
    Word/Concept & Northern & Central & Southern & Western \\\hline
    %grandmother    &\textit{ makom} &\textit{ makoma}  & \textit{ naa}&\textit{ naa} \\
    dance    &\textit{yag} &\textit{seɛ}  & \textit{sɛo}&\textit{sɛo} \\
    footwear    &\textit{nafag} &\textit{nɔɔteɛ}  & \textit{nageteɛ}&\textit{nataba} \\
    grandfather    &\textit{saangkom} &\textit{saangkoma}  & \textit{nabaale}&\textit{nabaale} \\
    groundnuts    &\textit{simmie} &\textit{seŋkãa}& \textit{gyɛnee}&\textit{gyɛnee} \\
    horse    &\textit{wir} &\textit{wiri}  & \textit{wɛo}&\textit{ycho} \\
    man    &\textit{dɛb} &\textit{dɔɔ}  & \textit{dao}&\textit{dao} \\
    rat    &\textit{dɛrebaa} &\textit{dayuu}  & \textit{gbunno}&\textit{gbinti} \\
    run    &\textit{zɔ} &\textit{zo}  & \textit{zɔ}&\textit{gyɔ} \\
    salt  & \textit{nyaaro} &  \textit{nyɛnoŋ}/\textit{nyɛnoo}& \textit{yaaroŋ}     & \textit{yaaroŋ}  \\
    talk/speak    &\textit{ɛr} &\textit{yeli}  & \textit{yɛle}&\textit{yɛle} \\
    tree squirrel    &\textit{telocra} &\textit{lɔnnɔ/lanna}  & \textit{lanta}&\textit{anta} \\
    uncle (maternal)    &\textit{madɛb} &\textit{areba}  & \textit{aheba}&\textit{areba} \\
    \lspbottomrule
    \end{tabularx}
    \caption{Lexical variation across four dialect zones}
    \label{tab:dialects}
\end{table}

    
    
     

   


%\begin{itemize}\setlength\itemsep{0em}
%  \item salt:  \textit{ nyɛnoŋ} (), \textit{ nyɛnoo} (), \textit{yaaroŋ} ()
%\item maternal uncle: \textit{ areba} (), \textit{ madɛb} () 
%\item paternal uncle (older than one’s father):  \textit{ zokpoŋ} (), \textit{ saankpeɛmɛ} (), \textit{bakpɛŋ} ()
%\item younger than one’s father: \textit{ saanbile} (), \textit{ babile} ()
%\end{itemize} 

\noindent Further variation is found in the other direction: One lexical item may have different meanings in different dialects. We illustrate with the term \textit{nabaale} which has two different meanings in two different Southern dialects, Waale and Manlaale:


\begin{itemize}\item \textit{nabaale}: `grandfather' (Walee), `paternal uncle (older than one’s father)'  (Manlaale)
\end{itemize}

%“Nabaale” would be a particularly interesting example to illustrate dialectal ambiguities, as it is used to mean “grandfather” in Waale, but “paternal uncle” (older than one’s father) in Manlaale, both dialects being classified as “Southern Dagaare” by the Dagaare Language Commission. 

We also note that some entries retain earlier orthographical forms, even though in modern pronunciation a reduced form is current.  This is particularly present for weak vowels following \textit{ŋ}, as in \textit{bɛŋe}  `to sift', which is pronounced [bɛŋ].



We would like to acknowledge the prior lexicographic work on Dagaare, most notably \citet{Durand1953} and \citet{Bodomo2004cantonese}, both being works which we have consulted and have attempted to build upon.



\subsection*{The structure of the entries}

%There is a small, but high quality linguistic literature on Dagaare which will provide a basis for further investigation of the lexical and grammatical behavior of Dagaare.  


The entries in the dictionary are structured as follows.  Each entry is followed by (i) its phonetic transcription, including tones, (ii) its part of speech, (iii) its definition in English, (iv) one or more examples in Dagaare illustrating the use of the word and the corresponding translation(s), and (iv) any further inflectional or derivational forms.  

Due to the nature of this work, the grammatical information contained in the entries is limited.  In addition to the sketch grammar provided here, several grammars on Dagaare exist which describe the grammatical features of the language and other selected topics.  We encourage readers to consult \citet{Bodomo1997},  \citet{Bodomo2000}, \citet{Bodomo2004cantonese,Bodomo2004complex}, and \citet{Dakubu2005} for further information on the grammar of Dagaare.

%**have followed the grammars, more work needs to be done***







\paragraph*{Nouns} For nouns, the following number forms are given:

\begin{itemize}\setlength\itemsep{0em}
 \item \textit{ pl.} — Plural
\item \textit{ 2ndpl.} — Second Plural (also known as a Distributive Plural) 
\item \textit{ sg.} — Singulative
\end{itemize}

\paragraph*{Adjectives}

Adjectives inflect for number in Dagaare, thus  plural and second plural forms are listed when present.  

\paragraph*{Pronouns}

Pronouns list plural forms, although due to their common occurrence, the plural forms of pronouns are also given their own main entry. 

\paragraph*{Verbs}

The verbal entries specify the following six forms at the end of the entry  in this order:  

\begin{itemize}\setlength\itemsep{0em}
\item Perfect \item Imperfect \item Agentive Singular \item Agentive Plural \item Agentive Derivation \textit{ -aa} Singular \item Agentive Derivation \textit{ -aa} Plural
\end{itemize}


These forms do not exhaust the derived verbal forms available in Dagaare, such as various nominalizations or derived adjectives, which we plan to incorporate in a future edition of this dictionary.

\paragraph*{Other parts of speech}

Adverbs, conjunctions, and interjections are indeclinable in Dagaare, so only the single form is listed along with its gloss.







\printbibliography[heading=subbibliography]
\end{refsection}
